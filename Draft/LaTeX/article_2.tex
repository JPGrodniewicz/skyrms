%%%%%%%%%%%%%%%%%%%%%%%%%%%%%%%%%%%%%%%%%
% Journal Article
% LaTeX Template
% Version 1.3 (9/9/13)
%
% This template has been downloaded from:
% http://www.LaTeXTemplates.com
%
% Original author:
% Frits Wenneker (http://www.howtotex.com)
%
% License:
% CC BY-NC-SA 3.0 (http://creativecommons.org/licenses/by-nc-sa/3.0/)
%
%%%%%%%%%%%%%%%%%%%%%%%%%%%%%%%%%%%%%%%%%

%----------------------------------------------------------------------------------------
%	PACKAGES AND OTHER DOCUMENT CONFIGURATIONS
%----------------------------------------------------------------------------------------

% \documentclass[twoside]{scrartcl}
\documentclass{article}

\usepackage{lipsum} % Package to generate dummy text throughout this template
\usepackage{graphicx}
\usepackage{epstopdf}

\usepackage[sc]{mathpazo} % Use the Palatino font
\usepackage[T1]{fontenc} % Use 8-bit encoding that has 256 glyphs
%\linespread{1.05} % Line spacing - Palatino needs more space between lines
\usepackage{microtype} % Slightly tweak font spacing for aesthetics

% amsmath package, useful for mathematical formulas
\usepackage{amsmath}
% amssymb package, useful for mathematical symbols
\usepackage{amssymb}
\usepackage[hmarginratio=1:1,top=32mm,columnsep=20pt]{geometry} % Document margins
\usepackage{multicol} % Used for the two-column layout of the document
%\usepackage[hang, small,labelfont=bf,up,textfont=it,up]{caption} % Custom captions under/above floats in tables or figures
\usepackage{booktabs} % Horizontal rules in tables
\usepackage{float} % Required for tables and figures in the multi-column environment - they need to be placed in specific locations with the [H] (e.g. \begin{table}[H])
\usepackage{hyperref} % For hyperlinks in the PDF
\usepackage{subcaption}

\usepackage{lettrine} % The lettrine is the first enlarged letter at the beginning of the text
\usepackage{paralist} % Used for the compactitem environment which makes bullet points with less space between them

\usepackage{titlesec} % Allows customization of titles
\renewcommand\thesection{\Roman{section}} % Roman numerals for the sections
\renewcommand\thesubsection{\Roman{subsection}} % Roman numerals for subsections
\renewcommand{\thetable}{S\arabic{table}}
\renewcommand{\thefigure}{S\arabic{figure}}
\titleformat{\section}[block]{\Large\scshape\centering}{\thesection.}{1em}{} % Change the look of the section titles
\titleformat{\subsection}[block]{\large}{\thesubsection.}{1em}{} % Change the look of the section titles

\usepackage{fancyhdr} % Headers and footers
\pagestyle{fancy} % All pages have headers and footers
\fancyhead{} % Blank out the default header
\fancyfoot{} % Blank out the default footer
\fancyhead[C]{Communication and Common Interest: Supporting Information} % Custom header text
\fancyfoot[RO,LE]{\thepage} % Custom footer text

\usepackage[numbers]{natbib}
\bibliographystyle{apalike}

% Remove brackets from numbering in List of References
%\makeatletter
%\renewcommand{\@biblabel}[1]{\quad#1.}
%\makeatother

\def\mathbi#1{\textbf{\em #1}} %bold italics in math definitions
\usepackage{flafter}



%----------------------------------------------------------------------------------------
%	TITLE SECTION
%----------------------------------------------------------------------------------------

\title{Communication and Common Interest\\ \Large{Supporting Information}} % Article title

\author{
\large
\textsc{Peter Godfrey-Smith, Manolo Mart\'inez}\\
\normalsize Philosophy Program, The Graduate Center, City University of New York\\ % Your institution
\normalsize mmartinez3@gc.cuny.edu
}


%----------------------------------------------------------------------------------------

\begin{document}

\maketitle % Insert title

%\thispagestyle{fancy} % All pages have headers and footers


%----------------------------------------------------------------------------------------
%	ARTICLE CONTENTS
%----------------------------------------------------------------------------------------

%\begin{multicols}{2} % Two-column layout throughout the main article text

\section*{Methods}

The games analyzed in the text were generated randomly from the
space of games with 3 equiprobable states, 3 messages, and 3
receiver actions. Payoffs were constrained to lie between 0 and 99.
The sampling was carried out using the \texttt{randrange} function in the
\texttt{random} module that is part of the Python standard library (version
3.3.2). The source code for this and all other calculations in this paper
is available at \href{https://github.com/manolomartinez/common-interest}{https://github.com/manolomartinez/common-interest}.

Formulas for the calculation of $C$, $C^*$, $K_S$, $K_R$, $K^*_S$, and $K^*_R$ are given below. Note that tied payoffs are handled in the following way.
When one agent has tied payoffs for two acts in a state, this is treated
as a concordant pair in the calculation of the $C$ parameters regardless of how the
other agent orders those acts in that state. The treatment of ties in
the calculation of the $K$ parameters is analogous.

\section*{Measures of Common Interest and Contingency of Payoff}

\begin{description}
    \item[$\mathbi{C}$:]
A measure of common interest between sender and receiver in a signaling
game in which payoffs for both players depend on the pairing of the
receiver's action with the state of the world. $C$ measures the extent to
which sender and receiver agree on their preference orderings over
actions in each state.
\end{description}

For states $\left\{S_{1},\dots,S_{n}\right\}$ and acts
$\left\{A_{1},\dots,A_{n}\right\}$, where the sender's payoff for act
$A_j$ in state $S_i$ is $\alpha_{ij}$, the receiver's payoff for act
$A_j$ in state $S_i$ is $\beta_{ij}$, we define the function
$D_{sender}$, that takes every triple of a state and two different acts
to a rational:

\[D_{sender}\left(S_i, A_j, A_k\right) = \begin{cases}
    0 & \mbox{if } \alpha_{ij} > \alpha_{ik} \\
    1 & \mbox{if } \alpha_{ij} < \alpha_{ik} \\
    \frac{1}{2} & \mbox{if } \alpha_{ij} = \alpha_{ik}
    \end{cases}\]

$D_{receiver}$ is defined analogously:

\[D_{receiver}\left(S_i, A_j, A_k\right) = \begin{cases}
    0 & \mbox{if } \beta_{ij} > \beta_{ik} \\
    1 & \mbox{if } \beta_{ij} < \beta_{ik} \\
    \frac{1}{2} & \mbox{if } \beta_{ij} = \beta_{ik}
    \end{cases}\]

Finally,

%\[C = 1 - \frac{2}{n(n-1)}\cdot \sum\limits_{1 \leq i \leq n}\sum\limits_{1 < j \leq n} \sum\limits_{1 \leq k < j} P\left(S_{i}\right)\cdot\left\lfloor D_{sender}(i, j, k) - D_{receiver}(i,j,k)\right\rfloor\]
\[C = 1 - \frac{2}{n(n-1)}\cdot \sum\limits_{\substack{1 \leq i \leq n, \\ 1 < j \leq n, \\ 1 \leq k < j}} P\left(S_{i}\right)\cdot\left\lfloor D_{sender}(i, j, k) - D_{receiver}(i,j,k)\right\rfloor\]

Some remarks regarding notation, which apply to all summations in this document:
\begin{itemize}

\item $\left\lfloor a \right\rfloor$ is the largest integer less than or equal to the absolute value of \emph{a}. 

\item $\sum\limits_{\substack{a, \\ \ldots \\ z}}$ is to be read as $\sum\limits_{a}  \ldots \sum\limits_{ z}$. 

\item In this summation, $k$ is always less than $j$. The intended effect is that when, e.g., $n=3$, then $\langle j,k \rangle$ takes the values $\langle 2,1 \rangle$, $\langle 3,1 \rangle$, $\langle 3,2 \rangle$. 

    \end{itemize}


\begin{description}
    \item[$\mathbi{C}^\mathbi{*}$:]
A finer-grained measure of common interest. $C^*$
measures the extent to which sender and receiver agree on their
preference orderings over actions in each state, but tracks also whether
sender and receiver agree that a given act in a state yields a payoff
higher than the (unweighted) mean of the payoffs that the agent can
receive in that state.
\end{description}

For each state $S_{i}$ we introduce an extra payoff value for both
sender and receiver:

\[\alpha_{in^\prime} = \frac{1}{n}\sum\limits_{1 \leq j \leq n} \alpha_{ij}\]
\[\beta_{in^\prime} = \frac{1}{n}\sum\limits_{1 \leq j \leq n} \beta_{ij}\]

where $n^\prime = n + 1$. Then,

%\[C^* = 1 - \frac{2}{n^\prime(n^\prime-1)}\cdot \sum\limits_{1 \leq i \leq n} \sum\limits_{1 < j \leq n^\prime} \sum\limits_{1 \leq k < j} P\left(S_{i}\right)\cdot\left\lfloor D_{sender}(i, j, k) - D_{receiver}(i,j,k)\right\rfloor\]
\[C^* = 1 - \frac{2}{n^\prime(n^\prime-1)}\cdot \sum\limits_{\substack{1 \leq i \leq n, \\ 1 < j \leq n^\prime, \\ 1 \leq k < j}} P\left(S_{i}\right)\cdot\left\lfloor D_{sender}(i, j, k) - D_{receiver}(i,j,k)\right\rfloor\]

\begin{description}
    \item[$\mathbi{K}_\mathbi{S}$ and $\mathbi{K}_\mathbi{R}$:]
$K_S$ ($K_R$) measures the extent to
which the sender's (receiver's) preference ordering over receiver actions 
varies with the state of the world.
\end{description}

With $D_{sender}$ and $D_{receiver}$ as above,

%\[K_{S} = \frac{2}{n(n-1)}\cdot \sum\limits_{1 < i \leq n} \sum\limits_{ 1 \leq j < i} \sum\limits_{ 1 < k \leq n} \sum\limits_{ 1 \leq l < k} \frac{2}{n(n-1)}\cdot\left\lfloor D_{sender}(i, k, l) - D_{sender}(j,k,l)\right\rfloor\]
\[K_{S} = \sum\limits_{\substack{1 < i \leq n, \\ 1 \leq j < i, \\ 1 < k \leq n, \\ 1 \leq l < k}} \frac{2}{n(n-1)}\cdot\left\lfloor D_{sender}(i, k, l) - D_{sender}(j,k,l)\right\rfloor\]

%\[K_{R} = \frac{2}{n(n-1)}\cdot \sum\limits_{1 < i \leq n} \sum\limits_{ 1 \leq j < i} \sum\limits_{ 1 < k \leq n} \sum\limits_{ 1 \leq l < k} \frac{2}{n(n-1)}\cdot\left\lfloor D_{receiver}(i, k, l) - D_{receiver}(j,k,l)\right\rfloor\]
\[K_{R} = \sum\limits_{\substack{1 < i \leq n, \\ 1 \leq j < i, \\ 1 < k \leq n, \\ 1 \leq l < k}} \frac{2}{n(n-1)}\cdot\left\lfloor D_{receiver}(i, k, l) - D_{receiver}(j,k,l)\right\rfloor\]

These formulas are not normalized. In the generation of data for this paper, the normalization coefficients (i.e., the maximum values for $K_S$ and $K_R$) were found numerically.

\begin{description}
    \item[$\mathbi{K}^*_\mathbi{S}$ and $\mathbi{K}^*_\mathbi{R}$:]
These are modifications of measures $K_S$ and $K_R$. The modification is strictly analogous to the one that takes $C$ to $C^*$.

\end{description}

With $D_{sender}, D_{receiver}, n^\prime, \alpha_{in^\prime}$, $\beta_{in^\prime}$ as above,

%\[K^{*}_{S} = \frac{2}{n^\prime(n^\prime-1)}\cdot \sum\limits_{1 < i \leq n} \sum\limits_{ 1 \leq j < i} \sum\limits_{ 1 < k \leq n^\prime} \sum\limits_{ 1 \leq l < k} \frac{2}{n(n-1)}\cdot\left\lfloor D_{sender}(i, k, l) - D_{sender}(j,k,l)\right\rfloor\]
\[K^{*}_{S} = \sum\limits_{\substack{1 < i \leq n, \\ 1 \leq j < i, \\ 1 < k \leq n^\prime, \\ 1 \leq l < k}} \frac{2}{n(n-1)}\cdot\left\lfloor D_{sender}(i, k, l) - D_{sender}(j,k,l)\right\rfloor\]

%\[K^{*}_{R} = \frac{2}{n^\prime(n^\prime-1)}\cdot \sum\limits_{1 < i \leq n} \sum\limits_{ 1 \leq j < i} \sum\limits_{ 1 < k \leq n^\prime} \sum\limits_{ 1 \leq l < k} \frac{2}{n(n-1)}\cdot\left\lfloor D_{receiver}(i, k, l) - D_{receiver}(j,k,l)\right\rfloor\]
\[K^{*}_{R} = \sum\limits_{\substack{1 < i \leq n, \\ 1 \leq j < i, \\ 1 < k \leq n^\prime, \\ 1 \leq l < k}} \frac{2}{n(n-1)}\cdot\left\lfloor D_{receiver}(i, k, l) - D_{receiver}(j,k,l)\right\rfloor\]

Again here, these formulas are not normalized. As for $K_S$ and $K_R$, the normalization coefficients were found numerically.
%------------------------------------------------

\section*{Additional Examples}

The games represented in Tables 2 and 3 of the main text are modeled on
games appearing in our computer-generated samples. Above are two
games from that sample, with $C=0$ (Table S1a) and $C^*=0$ (Table S1b).

\begin{table}
    \centering
    \begin{subtable}{.4\textwidth}
        %\centering
        \begin{tabular}{|c||c|c|c|}
        & $S_{1}$ & $S_{2}$ & $S_{3}$ \\
        \hline
        $A_{1}$ & 44,9 & 21,69 & 84,49 \\
        $A_{2}$ & 14,71 & 85,4 & 38,56 \\
        $A_{3}$ & 25,16 & 35,44 & 83,55 \\
        \end{tabular}
        \newline
        \flushleft{\emph{Sender}: $S_1 \rightarrow m_2; S_2 \rightarrow [(3/65)m_1,(62/65)m_2]; S_3 \rightarrow m_1$}
        \flushleft{\emph{Receiver}: $m_1 \rightarrow A_3; m_2 \rightarrow [(25/32)A_1, (7/32)A_2]; m_3 \rightarrow [(25/32)A_1, (7/32)A_2]$}
        \newline
        \caption{A game with $C=0$}
    \end{subtable}
    \begin{subtable}{.4\textwidth}
        %\centering
        \begin{tabular}{|c||c|c|c|}
        & $S_{1}$ & $S_{2}$ & $S_{3}$ \\
        \hline
        $A_{1}$ & 31,7 & 0,95 & 57,26 \\
        $A_{2}$ & 5,71 & 99,1 & 15,62 \\
        $A_{3}$ & 17,66 & 62,23 & 28,48 \\
        \end{tabular}
        \newline
        \flushleft{\emph{Sender}: $S_1 \rightarrow m_1; S_2 \rightarrow [(29/47)m_1,(18/47)m_2]; S_3 \rightarrow m_2$}
        \flushleft{\emph{Receiver}: $m_1 \rightarrow A_3; m_2 \rightarrow [(37/99)A_1, (62/99)A_2]; m_3 \rightarrow [(37/99)A_1, (62/99)A_2]$}
        \newline
        \caption{A game with $C^*=0$}
    \end{subtable}
    \caption{Games and their information-using equilibria}
    \label{tab:label}
\end{table}

In both these games -- and in contrast to the games in Tables 2 and 3 -- it
is \emph{not} the case that either sender or receiver, merely by deploying
pure strategies, can eliminate information use.  Once in an information-using equilibrium, no possible sequence of pure-strategy best responses by sender or receiver can take either of them to a strategy in which there is complete pooling (where A is a best response to B if and only if there is no response to B that gains a higher payoff than A). So no amount of ``drift'' in these cases can take sender and receiver from a situation in which information is used to a situation in which it is not.


The game with lowest common interest uncovered in our sample
in which there is an information-using equilibrium where neither
sender or receiver uses a mixed strategy is a game in which $C=0.22$
and $C^*=0.11$. This game is in Table S2a. Table S2b gives
another example discussed in the text, a case where common interest
is high ($C=0.78$) and mutual information between states and acts is
present at an equilibrium, but the amount of mutual information is very small: 0.03 bits.

\begin{table}
    \centering
    \begin{subtable}{.5\textwidth}
        %\centering
        \begin{tabular}{|c||c|c|c|}
        & $S_{1}$ & $S_{2}$ & $S_{3}$ \\
        \hline
        $A_{1}$ & 17,27 & 22,72 & 87,1 \\
        $A_{2}$ & 38,15 & 75,16 & 70,38 \\
        $A_{3}$ & 18,29 & 8,45 & 81,12 \\
        \end{tabular}
        \newline
        \flushleft{\emph{Sender}: $S_1 \rightarrow m_2; S_2 \rightarrow m_3; S_3 \rightarrow m_3$}
        \flushleft{\emph{Receiver}: $m_1 \rightarrow A_3; m_2 \rightarrow A_3;$\\$m_3 \rightarrow A_1$}
        \newline
        \caption{A game with $C=0.22$ ($C*=0.11$)\\ and a pure-strategy information-using\\ equilibrium}
    \end{subtable}
    \begin{subtable}{.4\textwidth}
        %\centering
        \begin{tabular}{|c||c|c|c|}
        & $S_{1}$ & $S_{2}$ & $S_{3}$ \\
        \hline
        $A_{1}$ & 92,79 & 23,59 & 99,64 \\
        $A_{2}$ & 68,57 & 40,80 & 22,86 \\
        $A_{3}$ & 7,41 & 23,66 & 28,64 \\
        \end{tabular}
        \newline
        \flushleft{\emph{Sender}: $S_1 \rightarrow m_2; S_2 \rightarrow m_3; S_3 \rightarrow m_2$}
        \flushleft{\emph{Receiver}: $m_1 \rightarrow A_3; m_2 \rightarrow [(6/77)A_1, (71/77)A_2]; m_3 \rightarrow A_2$}
        \newline
        \caption{A game with $C=0.78$ and mutual information between states and acts of 0.03 bits.}
    \end{subtable}
    \caption{Games and their information-using equilibria}
    \label{tab:label}
\end{table}


%------------------------------------------------

\section*{$C$, $C^*$, and constant-sum games}

Our approach to measuring common interest attends to the relations
between the sender's and receiver's preference orderings over actions
in each state (see above for details). $C$
assumes only ordinal utilities, while $C^*$ assumes cardinal utilities but
does not require that sender and receiver payoffs be commensurable.
An alternative concept of complete conflict of interest uses the idea of
a constant-sum game: a game is constant-sum if for each state there is
 a total payoff available that is divided in different ways between the
two agents, depending on the receiver's action. Actions affect
the division but not the amount divided. This concept of conflict of
interest assumes cardinal utilities and commensurability across
agents. This commensurability condition is not met in some
biologically relevant cases, such as those in which signaling occurs
across partners from different species -- often, different
kingdoms -- and these are cases where the concept of common interest
can be fruitfully applied \citep{Bergstrom2001, FitzGibbon1988, Harrison2005}

When the stronger set of assumptions required to classify
games as constant-sum, or not, are met, the following implications
hold. All constant-sum games have $C=0$, but not conversely. All $C^*=0$
games have $C=0$, but not conversely. Some constant-sum games are
$C^*=0$ and some are not, as a consequence of the role played by ties
(see above; when sender and receiver in a constant-sum three-act
game agree, for example, that their middle score is midway between
their high and low scores in a state, this gives them non-zero $C^*,$ as
was pointed out to us by Elliott Wagner). Lastly, not all $C^*=0$ games
are constant-sum.

As noted in the text, the function relating the proportion of
games with an information-using equilibrium to $C^*$ is similar to the
function for $C$. Figure S1 below, analogous to Figure 1 in the main text,
gives the proportion of games at each level of $C^*$ with at least one
 information-using equilibrium.

\begin{figure}[!ht]
\begin{center}
    \includegraphics{FigS1.eps}
\end{center}
\caption{The proportion of games at each level of $C^*$ with
at least one information-using equilibrium. For each value of $C^*$,
$n=1500$.}
\label{Figure_label}
\end{figure}

\section*{Interactions between common interest and \\contingency of
payoff}

Figure 3 in the main text shows interactions between common
interest ($C$) and contingency of payoff for sender ($K_S$) and receiver
($K_R$) . As expected, the proportion of games with information-using
equilibria generally increases with $C$ and with both $K_S$ and $K_R$.
Contingency of payoff for sender has a more complex role, however,
as when the sender's contingency of payoff is very low, intermediate
values of $C$ present a local maximum in the proportion of games with
 information-using equilibria. The asymmetry between the roles of
sender and receiver has the consequence that a sender can benefit
from sending informative signals even when $K_S=0$, as they need to
influence a receiver who seeks to vary their act with the state of the
world.

We performed a finer-grained analysis of these relationships by
using $C^*$ rather than $C$, and $K^*_S$ and $K^*_R$. These latter measures are
similar to $K_S$ and $K_R$ but are adjusted in the same way that $C^*$ adjusts
C. $K^*_S$, for example, compares the preference orderings over acts for
the sender in different states of the world, but also compares the
payoff for each act to the mean of the different payoffs the sender might receive in that
state. (Computational formulas are above.) The aim here, as in $C^*$, is
to track cases where two states share a second-best act whose payoff
is close to the best possible payoff in those states. The results are
given in Figure S2 below. The resulting analysis gives similar results to
those discussed in the text, with a somewhat clearer ``hump'' in the
sender's chart at low values of $K^*_S$ and intermediate values of $C^*$.

\begin{figure}[!ht]
\begin{center}
    \includegraphics[scale=0.9]{FigS2.eps}
\end{center}
\caption
    {Relation between common interest (using $C^*$)
contingency of payoff for each agent (using $K^*_S$ and
$K^*_R$), and the proportion of games with
an information-using equilibrium. 1500 games were sampled and analyzed for each
jointly possible combination of $C^*$ and $K^*_S$ ($K^*_R$).}
\label{Figure_label}
\end{figure}


%------------------------------------------------
\bibliography{/home/manolo/.pandoc/default}
%----------------------------------------------------------------------------------------

%\end{multicols}

\end{document}
