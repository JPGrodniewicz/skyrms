% Template for PLoS
% Version 1.0 January 2009
%
% To compile to pdf, run:
% latex plos.template
% bibtex plos.template
% latex plos.template
% latex plos.template
% dvipdf plos.template

\documentclass[10pt]{article}

% amsmath package, useful for mathematical formulas
\usepackage{amsmath}
% amssymb package, useful for mathematical symbols
\usepackage{amssymb}

% graphicx package, useful for including eps and pdf graphics
% include graphics with the command \includegraphics
\usepackage{graphicx}
\usepackage{epstopdf}

% cite package, to clean up citations in the main text. Do not remove.
\usepackage{cite}

\usepackage{color} 

% Use doublespacing - comment out for single spacing
\usepackage{setspace} 
\doublespacing


% Text layout
\topmargin 0.0cm
\oddsidemargin 0.5cm
\evensidemargin 0.5cm
\textwidth 16cm 
\textheight 21cm

% Bold the 'Figure #' in the caption and separate it with a period
% Captions will be left justified
\usepackage[labelfont=bf,labelsep=period,justification=raggedright]{caption}

% Use the PLoS provided bibtex style
\bibliographystyle{plos2009}

% Remove brackets from numbering in List of References
\makeatletter
\renewcommand{\@biblabel}[1]{\quad#1.}
\makeatother


% Leave date blank
\date{}

\pagestyle{myheadings}
%% ** EDIT HERE **


%% ** EDIT HERE **
%% PLEASE INCLUDE ALL MACROS BELOW

%% END MACROS SECTION

\begin{document}

% Title must be 150 characters or less
\begin{flushleft}
{\Large
\textbf{Communication and Common Interest}
}
% Insert Author names, affiliations and corresponding author email.
\\
Peter Godfrey-Smith$^{1}$, 
Manolo Mart\'inez$^{2, \ast}$
\\
\bf{1} Philosophy Program, The Graduate Center CUNY, New York City, NY, USA
\\
\bf{2} Philosophy Program, The Graduate Center CUNY, New York City, NY, USA
\\
$\ast$ E-mail: mmartinez3@gc.cuny.edu
\end{flushleft}

% Please keep the abstract between 250 and 300 words
\section*{Abstract}

Explaining the maintenance of communicative behavior in the face of
incentives to deceive, conceal information, or exaggerate is an
important problem in behavioral biology. When the interests of agents
diverge, some form of signal cost is often seen as essential to
maintaining honesty. Here, novel computational methods are used to investigate the role of
common interest between the sender and receiver of messages in
maintaining cost-free informative signaling in a signaling game. Two
measures of common interest are defined. These quantify the divergence
between sender and receiver in their preference orderings over acts the
receiver might perform in each state of the world. Sampling from a large
space of signaling games finds that informative signaling is possible at
equilibrium with zero common interest in both senses. Games of this kind
are rare, however, and the proportion of games that include at least one
equilibrium in which informative signals are used increases
monotonically with common interest. Common interest as a predictor of
informative signaling also interacts with the extent to which agents'
preferences vary with the state of the world. Our findings provide a quantitative description of the relation between
common interest and informative signaling, employing exact measures of
common interest, information use, and contingency of payoff under
environmental variation that may be applied to a wide range of models
and empirical systems.

% Please keep the Author Summary between 150 and 200 words
% Use first person. PLoS ONE authors please skip this step. 
% Author Summary not valid for PLoS ONE submissions.   
\section*{Author Summary}

How can honest communication evolve, given the many incentives to
deceive, conceal information, or exaggerate? In recent work, it has
often been supposed that either common interest between the sender and
receiver of messages must be present, or special factors (such as a
special cost for dishonest production of signals) must be in place. When
talk is cheap, what is the minimum degree of common interest that will
suffice to maintain communication? We give new quantitative measures of common interest between
communicating agents, and then use a computer search of signaling games
to work out the relationship between the degree of common interest and
the maintenance of signaling that conveys real information.
Surprisingly, we find that informative signaling can in some cases be
maintained with zero common interest. These cases are rare, and we also
find that the degree of common interest is a good predictor of whether
informative signaling is a likely outcome of an interaction. The upshot is that two agents with highly incompatible preferences may
still find ways to communicate, but the more they see eye-to-eye, the
more likely it is that communication will be viable.

\section*{Introduction}

Many theorists have seen communication as a fundamentally cooperative
phenomenon \cite{Lewis1969, Grice1975, Millikan84, Tomasello2008}. In
an evolutionary context, however, cooperation cannot be taken for
granted, because of problems of subversion and free-riding
\cite{Williams1966}. In the case of communication, these problems
include both refusal to share information, and deception, or lying for
one's own advantage. If lying is common, there is no point in listening
to what anyone says. If no one is listening, there is no point in
talking.

In recent work the situation is often sketched as follows: it is easy to
see how communication can be viable if there is complete concordance of
interests between senders and receivers of signs. Then communication can
result in useful coordination and division of labor. There is no mystery
about signaling within multicellular organisms, for example, including
hormonal and cell-to-cell signaling (although conflicts of interest may
arise even here: \cite{Haig2008}). In between-organism contexts, the
problem of conflict of interest rapidly becomes acute. Special
mechanisms are needed to explain how honesty is maintained. The main
approach taken in recent years has been \emph{costly signaling theory}
\cite{Zahavi1975, Grafen1990, Maynard-Smith2003}. Intrinsic costs of
signaling prevent dishonesty, by differential expense to liars or
differential benefits to the honest.

``Cheap talk'' models, where signaling has no costs, have seen some
development
\cite{Crawford1982, Farrell1996, Bergstrom1998, Silk2000, Bradbury2000, Wagner2012}
but have been minor players in recent years. Here we use a novel method
to examine ways that informative signaling can be sustained without cost
in a range of situations of partial and low common interest. We use a
version of the Lewis sender-receiver model
\cite{Lewis1969, Skyrms2010}, and employ a method of sampling and
analyzing cases drawn from a large space of games with different
relationships between sender and receiver payoffs. We then offer
generalizations based on analysis of the sample of cases. The analysis
uses coarse-grained measures of common interest between sender and
receiver, and attends also to a feature that interacts with common
interest: the degree to which payoffs for an agent depend on different
acts being produced in different states, the \emph{contingency of
payoff} for that agent.

We find that using a simple and intuitive measure of common interest
based on comparisons of preference orderings over actions, it is
possible, though rare, for informative signaling to be maintained at
equilibrium with complete divergence of interests. We then construct a
more fine-grained measure of common interest, one that is more demanding
in its classification of a case as one of zero common interest, and find
that informative signaling with zero common interest is possible in this
stronger sense as well. Defining an \emph{information-using equilibrium}
as one where the receiver makes use of informative signals to guide
behavior, the proportion of games that include at least one
information-using equilibrium increases monotonically and rather
smoothly with both measures of common interest. (See below, in the \emph{Methods} section, for the equilibrium concept we use throughout the paper.) We then look at the
equilibria that support the \emph{highest} amount of information use for
a given level of common interest, and again find a monotonic, though
less smooth, relationship between degree of common interest and maximum
information use. A third analysis, looking at the relationship between
common interest and contingency of payoff for sender and receiver
(defined below), yields more complicated results.

We conclude that informative signaling can be stable in situations of
minimal, even zero, common interest. A combination of \emph{mixed
strategies} of signal use by both senders and receivers, and the
selective \emph{pooling} of states by the sender, makes possible the
extreme cases of this phenomenon. Pooling alone can suffice in cases
where divergence of interests is not so extreme. As interests converge,
stability of informative signaling becomes easier to achieve. Our model
complements other recent work on the adaptive importance of mixed
strategies and partially informative signaling in evolution.


% Results and Discussion can be combined.
% You may title this section "Methods" or "Models". 
% "Models" is not a valid title for PLoS ONE authors. However, PLoS ONE
% authors may use "Analysis" 
\section*{Methods}

Our modeling framework draws on Lewis \cite{Lewis1969} and Skyrms
\cite{Skyrms2010}. We assume that the world varies exogenously and has
three equally probable states ($S_{1}$, $S_{2}$,
$S_{3}$). The sender perceives (without error) the state of
the world and responds by mapping states to signals ($m_{1}$,
$m_{2}$, $m_{3}$). The mapping need not be
one-to-one as the sender may ``pool'' some states, treating them
equivalently, and the sender may also probabilistically ``mix'' signals
in response to a given state. The receiver perceives (without error) the
signal sent and maps signals to acts ($A_{1}$,
$A_{2}$, $A_{3}$), with pooling and mixes possible
again. So a combination of sender and receiver rules can be represented
as follows:

\begin{description}
\item[Sender:]
$S_1\rightarrow m_1;S_2\rightarrow m_2;S_3\rightarrow \left[(2/3)m_1, (1/3)m_3\right]$
\item[Receiver:]
$m_1\rightarrow A_1; m_2\rightarrow \left[(1/2)A_1, (1/2)A_2\right]; m_3\rightarrow A_2$
\end{description}

For example, the sender here sends message 1 whenever they see state 1,
message 2 whenever they see state 2, and in state 3 they flip a biased
coin to send message 1 two thirds of the time and message 3 one third of
the time. Both sides receive payoffs as a consequence of the combination
of the receiver's action and the state of the world. Sender and receiver
payoffs may differ, and can be represented in the form seen in Table 1.

The payoff matrix defines a preference ordering over acts in each state
for both sender and receiver. For example, in Table 1, the preference
ordering for the sender in state 1 is {[}$A_{2}$
\textgreater{} $A_{1}$ \textgreater{} $A_{3}${]},
and for the receiver {[}$A_{3}$ \textgreater{}
$A_{2}$ \textgreater{} $A_{1}${]}. A simple
measure of the degree of common interest in a game tracks how similar
the orderings for sender and receiver are, for each state: there is
\emph{complete common interest} when sender and receiver have the same
preference ordering over acts in every state, and \emph{complete
conflict of interest} when these orderings are reversed in every state.
Between these extremes are various kinds of \emph{partial} common
interest: sender and receiver might agree on the best act in each state,
but disagree otherwise; they might always agree on what is worst, but
not otherwise; they might agree entirely in some states but disagree in
others.

In cases of complete common interest, some consequences for informative
signaling are easily seen. With complete common interest, sender and
receiver can both receive their maximum payoffs when the sender maps
states to signals one-to-one and the receiver uses these signals to
guide appropriate actions. This is a \emph{signaling system} in the
sense of Lewis \cite{Lewis1969}, and neither party has any incentive to change what
they are doing. This state might not be attained by the selection
process shaping sender and receiver behaviors, but if it is reached it
is stable \cite{Huttegger2010}. With complete conflict of interest, it would appear that
signaling cannot be maintained, as any information about the state of
the world carried by signals can be used by the receiver to produce acts
contrary to the sender's interests, and any sensitivity to signals in
the receiver can be exploited by the sender. Exploring the generality of
this phenomenon is one aim of this paper. Another is quantifying the
relationship between common interest and informative signaling.

The varieties of partial common interest described above do not form a
complete ordering. However, a coarse-grained measure of the overall
degree of common interest can be constructed by modifying the
\emph{Kendall tau distance}. This measure describes the similarity in
the ordering of the items in two lists, by counting \emph{discordant
pairs} of items across the lists. The first two items in the two lists
form a discordant pair with respect to a preference ordering, for
example, if in list 1 the first item is preferred to the second item,
whereas in list 2 the second item is preferred to the first. We define a
measure \emph{C} of the common interest in a payoff matrix of the form
in Table 1 by counting the discordant pairs in the sender's and
receiver's preference orderings over acts in each state of the world, and
then averaging across states and rescaling the results to yield a number
between 0 and 1, where $C=1$ corresponds to complete common interest and
$C=0$ corresponds to complete conflict of interest. In response to
results outlined below we also make use of a refinement of $C$; which
compares not only the agents' preference orderings of the actions in
each state, but also tracks how the agents' payoffs for each action relate to the mean
value of the payoffs the agent might receive in that state. (For details see Text S1.) As discussed below, $C^*$ is one
among several ways of refining the simpler measure, $C$, and we do not
claim it is best for all purposes.

We also make use of a further description of payoff matrices. For each
agent, how much does payoff depend on matching different actions to each
state of the world? A simple illustration of the importance of this
feature is seen in a case where the receiver has the same best act for
every state (has a dominant strategy available). Then the receiver can
achieve maximum payoff no matter what the sender does, by mapping all
signals to that cover-all act. Even if no one act is best in all
states, there may be a cover-all act that works well for an agent nearly
all the time. This is a within-agent matter. So we define $K_S$ and
$K_R$, also making use of the Kendall tau distance. For each agent, we
compare the preference orderings over acts that apply in different
states of the world, comparing each pair of states in turn. \emph{K} is
high for an agent with respect to a pair of states if good acts in one
state are bad acts in the other state. \emph{K} for an agent averages
all comparisons of states, rescaled to lie between zero and one, where
$K=1$ corresponds to the highest degree of contingency of payoff. (For
details see Text S1.)

Our aim is to generalize about games with different levels of common
interest and contingency of payoff for the agents. The method used is to
generate samples from the space of games with three states where sender
and receiver payoffs are integers between 0 and 99. Payoffs for each
player for each act in a state are chosen randomly, so 18 random choices
specify payoffs for a game. We then use the implementation of Lemke's
\cite{Lemke1965} algorithm provided by the software package \emph{Gambit} \cite{McKelvey2010} to
search for equilibria in that game where informative signals are being
sent and used. The equilibrium concept used is the Nash equilibrium: a
pair of strategies form a Nash equilibrium if neither player can improve
their payoff by unilaterally modifying their strategy.

We measure the degree to which agents engage in informative signaling
with \emph{mutual information}, a symmetrical measure of the degree of
association between two variables, measured in bits \cite[p.~7]{Cover2006}. An equilibrium is an
\emph{information-using} equilibrium if there is non-zero mutual
information between states of the world and the receiver's acts. We
focus on mutual information between states and acts for the following
reasons. If there is mutual information between states and acts, the
only way for this to arise is for senders to send informative signals
and receivers to use these signals to guide variation in their actions
to some extent. It is possible for senders to send signals with
information about the state of the world that is not used -- informative
signals that are ignored by the receiver. It is possible also for
receivers to guide actions with different signals sent randomly by the
sender. The first of these -- informative signals that are ignored -- is
a situation which may be an equilibrium and in which there is
informative signaling, but it is not a situation in which the receiver
is making use of that information. Our primary focus is situations in
which informative signals are both sent and used. This requires that the
signals carry information about states and acts carry information about
signals. Given that receivers only have access to the state of the world
by attending to signals, by the \emph{data processing inequality} \cite[p.~34]{Cover2006}
it is not possible for acts to carry more information about states than
signals do. (States, signals, and acts form a Markov chain.) Any mutual
information between states and acts arises from the use by the receiver
of information about states in the signals.

Computational methods are described in Text S1 but
one feature should be noted here: Lemke's algorithm is not guaranteed to
find every equilibrium in a game \cite{Koller1996}. So the reports of
information-using equilibria below may be under-counts.

\section*{Results}

To investigate the role of \emph{C} we generated a random sample from
the space of games with three equiprobable states, three receiver
actions, and independently chosen payoffs for sender and receiver
associated with each receiver action in each state of the world. (Each
value of \emph{C} is represented by 1500 games.) These sender and
receiver payoffs are integers between 0 and 99. For each game we asked
whether there is at least one information-using equilibrium in that game
-- an equilibrium with nonzero mutual information between states and
acts -- and then asked what proportion of games at each level of
\emph{C} have at least one information-using equilibrium. (All these
games also have equilibria that are not information-using equilibria).
The results are shown in Figure 1.

\emph{Figure 1 to be placed here.}

Very low degrees of \emph{C} suffice to enable information-using
equilibria, but at low \emph{C} levels, only a small minority of games
do so (unless the algorithm used has significant bias). As \emph{C}
increases, the fraction of games with information-using equilibria
increases monotonically.

The curve in Figure 1 does not reach 100\% for the case of complete
common interest. Some games with $C=1$ are games with zero
$K_{S}$ and $K_{R}$. (When $C=1$, \emph{K} is the
same for sender and receiver.) The same act is best in every state. Around 1/9 games with $C=1$ will also be $K=0$. In such a game, the
receiver can always take the system to an equilibrium by mapping all
signals to the same, optimal, act. Then there is no mutual information
between states and acts, regardless of what the sender is doing, as
there is no variation in acts.

Surprisingly, a small number of games with $C=0$, where sender and
receiver have reversed preference orderings over acts in every state,
have information-using equilibria. Table 2 shows a case of this kind --
not a case from one of our samples, but a simplified case constructed
using the computer-generated cases as a guide.

Despite zero $C$, the game in Table 2 has an information-using
equilibrium, whose sender and receiver rules are as follows:

\begin{description}
\item[Sender:]
$S_1\rightarrow m_1;S_2\rightarrow [(1/2)m_1, (1/2)m_3]; S_3\rightarrow m_3$
\item[Receiver:]
$m_1\rightarrow A_1; m_2\rightarrow [(2/3)A_2, (1/3)A_3]; m_3\rightarrow [(2/3)A_2, (1/3)A_3]$
\end{description}

The mutual information between states and acts at this equilibrium is
0.67 bits, where the highest possible value for a game with three
equiprobable states (a Lewisian signaling system) is 1.58 bits.

A feature of the case in Table 2 is that although sender and receiver
have reversed preferences in every state, in $S_1$ they share a second-best
outcome ($A_1$) that is almost as good as their best. This is ignored by
our measure $C$, and it is one kind of common interest between the two
agents. A way to modify \emph{C} that takes this factor into account is
to compare, across sender and receiver, their preference orderings over
both the payoffs that arise from different actions and also the average
of the payoffs for that agent in that state. This is done by defining a "dummy act" for the receiver in each state, an act that secures for each agent the mean of the other payoffs possible in that state. This dummy act and its payoff are then included in the determination of each agent's preference ordering over acts in that state; the two agents might agree, or disagree, for example, about whether the payoff of Act 1 is higher than the mean of their payoffs possible in that state. $C^*$, like $C$, counts discordant pairs of preferences and is scaled to lie between 0 and 1. (For further details see Text S1).  $C^*$ yields
a similar relationship between common interest and the proportion of
games with an information-using equilibrium to that seen in Figure 1.

The game in Table 2 has a nonzero $C^*$, as sender and receiver
agree about how one of their second-best outcomes compares to their
means for that state, so $C^*=0$ is a more demanding criterion for
complete conflict of interest. Even in this stronger sense, though, it
is possible for a game to have an information-using equilibrium with
complete conflict of interest. A case of this kind, also one modeled on
a less transparent computer-generated case, is shown in Table 3. This
game has the following information-using equilibrium:

\begin{description}
\item[Sender:]
$S_1\rightarrow m_3; S_2\rightarrow [(3/7) m_2, (4/7) m_3]; S_3\rightarrow m_2$
\item[Receiver:]
$m_1\rightarrow A_2; m_2\rightarrow [(5/7)A_1, (2/7)A_3]; m_3\rightarrow A_2$
\end{description}

In all the cases with $C=0$ and/or $C^*=0$ with information-using
equilibria we have found, the underlying pattern is as follows. Two
signals are used by the sender and three acts are used by the receiver.
In one state the receiver produces an act that is intermediate in value
for both sides. In the cases in Tables 2 and 3, this is $S_1$. The receiver is
prevented from shifting to their optimal act for this state by the fact
that the signal sent in that state is ambiguous, and is sometimes also
sent in a state for which the act that might ``tempt'' the receiver in
$S_1$ would be very bad. In another state, the receiver mixes their actions
between optimal acts for each side. (This is $S_3$ in both Tables
2 and 3.) Again, the receiver is prevented from settling on their optimal act
in $S_3$ by the fact that the message the sender sends in that state is
ambiguous; state 2 is used by the sender to deter exploitation in the
other two states, and in this state all three acts are produced.

In both cases in Tables 2 and 3 the information-using equilbria are very
fragile, as either the sender (in 3) or the receiver (in 2) can shift
without penalty to a strategy in which the mutual information between
states and acts goes to zero. Not all cases of information-using
equilbria and zero common interest have this feature, however; sometimes
information-use is less easily lost. The lowest level of common interest
at which an information-using equilibrium is found in which neither
sender nor receiver plays a mixed strategy, probabilistically varying
their response to a state or a signal, is $C^*=0.11$ (see Text S1 for examples of both phenomena described in this paragraph).

A valuable feature of \emph{C} is the weakness of the assumptions
required for its measurement; \emph{C} assumes only ordinal, not
cardinal, utilities. $C^*$ assumes cardinal utilities. $C^*$ does not,
however, assume that sender and receiver utilities are commensurable. If
that further assumption is made, the notion of zero common interest can
be analyzed instead by requiring that in every state, sender and
receiver payoffs sum to a constant and the choice of action determines
only how the division is made (a ``constant-sum game''). We do not claim
in this paper that information-using equilibria exist in constant-sum
games. All constant-sum games have $C=0$, though the converse does not
hold. Some constant-sum games have nonzero $C^*$, on the other hand, and
not all $C^*=0$ games are constant-sum. Due to its simplicity and weak
assumptions, in the remainder of the body of this paper we will use
\emph{C} to measure common interest. $C^*$ and constant-sum games are
discussed in Text S1.

Once we know how likely a given level of \emph{C} is to maintain at
least one information-using equilibrium, we can also ask what is the
highest level of mutual information between states and acts that can be
maintained in a game with a given degree of $C$. Figure 2 shows the
maximum amount of mutual information between states and acts generated
by an equilibrium pair of strategies from any game examined with a given
level of $C$. In constructing the pool of cases for this analysis, we have
included not just the sample of games used in Figure 1 but also games
found in earlier samples.

\emph{Figure 2 to be placed here.}

Figure 2 shows that the highest value for information use grows
monotonically with common interest, as expected, but in a step-like way
and with quite high values of mutual information between states and acts
seen even at the lowest values of $C$. Conversely, our sample includes
cases with high values of \emph{C} and very minimal information use at
equilibrium ($C=0.78$, mutual information = 0.03 bits; see Text S1).

A further analysis of these cases takes into account the contingency of
payoff for sender and receiver, as well as common interest. The
importance of this factor has been evident already in some extreme
cases. When there is complete common interest but \emph{K} is zero for
both sides, there is no problem for signaling to solve -- a single act
always delivers an optimal payoff. When there is less common interest,
the contingency of payoff for sender and receiver can diverge, and in
most cases will be different. Figure 3 charts the proportion of games
with at least one information-using equilibrium as a function of both
common interest and contingency of payoff for an agent; separate graphs
are given for $C$ and $K_S$ (left), and for $C$ and $K_R$
(right). The sample used for this chart is not the same one used for
Figure 1, as a random sample of all games under-represents games with
very low and very high values of $C$ and $K$. Figure 3 uses a sample
in which every combination of $C$ and $K$ is represented by
1500 games.

\emph{Figure 3 to be placed here.}

As expected, higher values of $K$ generate more information-using
equilibria than lower values of $K$. A difference is seen, however,
between the consequences of low values of $K_S$ and $K_R$. When the
sender's contingency of payoff is very low, the intermediate values of
$C$ present a local maximum in the proportion of games with
information-using equilibria. When $K_S$ is low and $C$ is
intermediate, $K_R$ will be appreciable. The receiver seeks to vary
their actions with the state of the world, and though the sender would
ideally like the same act to always be performed, equilibria exist in
which a compromise is reached. When the receiver's $K$ is low, on
the other hand, they can achieve optimal payoffs by mapping every signal
to the same act. The receiver can ``go it alone'' (though
information-using equilibria arise in a few cases with high $C$ 
because of ties for the optimal act in a state).

\section*{Discussion}

We have given a treatment of the relation between informative signaling
and common interest between sender and receiver, in a framework where
signal use is associated with no differential costs and no role is given
to iteration of interactions between agents. We find that informative
signaling is possible in situations where sender and receiver have
reversed preference orderings over receiver actions in every state of
the world. This situation, where $C=0$, is one sense of ``complete
conflict of interest,'' and a sense that has been employed more
informally in a range of earlier discussions (eg., \cite{Maynard-Smith1994,Searcy2005}. In the light of our results, $C=0$ is shown to
be a somewhat undemanding sense of complete conflict. We discussed one
refinement of $C$, which requires stronger assumptions about payoffs, and
found that information use at equilibrium is possible with complete
conflict even in this stronger sense, where $C^*=0$. Another way to
refine the idea of complete conflict, a way that uses still stronger
assumptions, is by appeal to the notion of a constant-sum game. We do
not claim that informative signaling is possible at equilibrium in
constant-sum games. Another way to interpret our results is to suggest
that the degree of conflict of interest in a game cannot be analyzed by
noting the relationships holding between preferences in particular
states, and then generalizing across states. Moving beyond consideration
of these extreme values, we find that $C$ is a good predictor of
the existence of information-using equilibria in the space of games
studied in this paper.

We note several limitations of our model. First, the model assumes a
particular relationship between sender and receiver, one where the
sender has private knowledge of a state of the world, and payoffs
result from the coordination of receiver actions with this state. This
``state'' of the world might be the condition or quality of the sender.
Another kind of model assumes that neither side has privileged
information about the state of the world, and the role of signaling is
to coordinate acts with acts rather than acts with states (the ``battle
of the sexes,'' for example). In further work we hope to extend our
analysis to cover these cases. Another limitation involves our use of
the Nash equilibrium concept. A Nash equilibrium need not be an
evolutionarily stable strategy (because rivals may increase in frequency
due to ``drift''). In addition, equilibria of this kind may not be
easily found by an evolutionary process \cite{Huttegger2010}. Further
work is needed to explore the dynamic properties of the games discussed
in this paper. Thirdly, our analysis gives no role to the biological
plausibility of games.

We close by comparing our treatment with two other papers, one classic
and one recent. First, Crawford and Sobel \cite{Crawford1982} treated agreement in
interests as a matter of degree, and found that when interests diverge,
honest signaling is possible, but with lower informational content than
there would be with complete agreement: ``equilibrium signaling is more
informative when agents' preferences are more similar.'' In their model,
the state of the world (sender quality) and the available actions both
vary continuously in one dimension, and the difference between sender
and receiver interests corresponds to a constant that is the difference
between the actions seen as optimal by sender and by receiver in a given
state of the world. In their model the degree of common interest across
games can be measured exactly, but the model makes strong assumptions
about the pattern of variation in the world. Our model makes weaker
assumptions in this area, with the consequence that common interest is
only partially ordered, motivating the introduction of coarse-grained
measures such as \emph{C} and $C^*$. Crawford and Sobel found that as
agents' interest converge, a larger number of distinct signals can be
sent at equilibrium. We found that informative signaling can exist with
zero common interest, through a combination of pooling and mixing,
though games of this kind are rare and the proportion of games with an
information-using equilibrium increases as interests converge. Crawford
and Sobel's model also did not allow for variation in $K$, which we find
has significant effects on the viability of information use.

Second, Zollman \emph{et al.} \cite{Zollman2013} investigated biologically plausible games
with two possible states of the world (again, sender quality) that are
usually analyzed with substantial differential costs enforcing honesty.
These authors found that very small differences in cost or benefit
across different types of senders can maintain honest signaling when
both sender and receiver mix strategies in a particular way. Senders in
one state mix two signals, and senders in another state send just one of
those signals. Receivers mix their responses to the ambiguous signal and
do not mix their responses to the other. A conclusion from their model
is that variation in signal-using behavior within a given situation, on
both sender and receiver sides, need not be a matter of mere ``noise''
but can be an essential feature of an equilibrium state. Our results,
within a framework of zero signal cost, lead to a conclusion of the same
kind: probabilistic mixing of strategies, along with partial ``pooling''
of inputs, by both sign producers and sign interpreters can be important
in maintaining signaling in situations of low common interest.

% Do NOT remove this, even if you are not including acknowledgments
\section*{Acknowledgments}

We are grateful to Carl Bergstrom, Simon Huttegger, Ron Planer, Gill
Shen, Rory Smead, Elliott Wagner, and Kevin Zollman for helpful comments
on an earlier draft.

%\section*{References}
% The bibtex filename
\bibliography{/home/manolo/.pandoc/default}

\pagebreak
\section*{Figure Legends}
\begin{figure}[!ht]
\begin{center}
%\includegraphics[width=4in]{Fig1.eps}
\end{center}
\caption{
{\bf The proportion of games at each level of \emph{C} with
at least one information-using equilibrium.} For each value of $C$,
$n=1500$.}
\label{Figure_label}
\end{figure}

\begin{figure}[!ht]
\begin{center}
%\includegraphics[width=4in]{Fig2.eps}
\end{center}
\caption{
{\bf The highest level of information use at each level of $C$.} Measured in bits. For each value of $C$, $n=1500$.}
\label{Figure_label}
\end{figure}

\begin{figure}[!ht]
\begin{center}
%\includegraphics[width=4in]{Fig3.eps}
\end{center}
\caption{
    {\bf Relation between common interest,
contingency of payoff for each agent, and the proportion of games with
an information-using equilibrium.} See Text S1 for explanations of
C, $K_S$ and $K_R$. 1500 games were sampled and analyzed for each
jointly possible combination of \emph{C} and $K_S$ ($K_R$).}
\label{Figure_label}
\end{figure}
\pagebreak

\section*{Tables}

\begin{table}[!ht]
\caption{
\bf{A Payoff Matrix}}
\begin{tabular}{|c||c|c|c|}
& $S_{1}$ & $S_{2}$ & $S_{3}$ \\
\hline
$A_{1}$ & 5,0 & 2,4 & 0,6 \\
$A_{2}$ & 6,5 & 0,0 & 1,5 \\
$A_{3}$ & 0,6 & 6,6 & 5,3 \\
\end{tabular}
\begin{flushleft}The pair of numbers in each cell represent the sender's and the
receiver's payoffs, respectively, for a receiver action ($A$) performed
in a given state of the world ($S$).
\end{flushleft}
\label{tab:label}
\end{table}

\begin{table}[!ht]
\caption{
\bf{A game with $C=0$ and an information-using equilibrium}}
\begin{tabular}{|c||c|c|c|}
& $S_{1}$ & $S_{2}$ & $S_{3}$ \\
\hline
$A_{1}$ & 5,5 & 2,4 & 2,1 \\
$A_{2}$ & 6,0 & 0,6 & 3,0 \\
$A_{3}$ & 0,6 & 6,0 & 0,3 \\
\end{tabular}
\begin{flushleft}
\end{flushleft}
\label{tab:label}
 \end{table}

\begin{table}[!ht]
\caption{
\bf{A game with $C^*=0$ and an information-using equilibrium.}}
\begin{tabular}{|c||c|c|c|}
& $S_{1}$ & $S_{2}$ & $S_{3}$ \\
\hline
$A_{1}$ & 1,8 & 8,1 & 0,6 \\
$A_{2}$ & 3,7 & 6,3 & 1,5 \\
$A_{3}$ & 8,1 & 1,8 & 5,3 \\
\end{tabular}
\begin{flushleft}
\end{flushleft}
\label{tab:label}
 \end{table}

\section*{Supporting Information Legend}
\begin{description}
    \item[Text S1:]Methods -- Definitions -- Additional Examples -- $C$, $C^*$, and constant-sum games -- Interactions between common interest and contingency of payoff
\end{description}
\end{document}

